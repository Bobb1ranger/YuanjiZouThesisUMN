%%%%%%%%%%%%%%%%%%%%%%%%%%%%%%%%%%%%%%%%%%%%%%%%%%%%%%%%%%%%%%%%%%%%%%%%%%%%%%%%
% app_glossary.tex: Glossary Appendix:
%%%%%%%%%%%%%%%%%%%%%%%%%%%%%%%%%%%%%%%%%%%%%%%%%%%%%%%%%%%%%%%%%%%%%%%%%%%%%%%%
\chapter{RMSS Appendices}
\label{app_glossary}
\section{System configuration}
\begin{table}[h]
    \caption{The setting of system parameters}\label{tab:sys_config}
    \centering
    \begin{tabular}{ccc}
    \hline
        Properties  & Value &description\\
    \hline
        $m_c$ & 2.5 kg& Mass of cart\\
        $m_p$ & 0.1 kg& Mass of pendulum\\
        $l$ & 2 m &Length of pendulum\\
        $\tau$ & 0.05 s&Sampling step time\\
        \hline
    \end{tabular}
\end{table}
Let $x=[\theta,\dot{\theta},\dot{s}]^T$, the upper settings yield following discrete time state space matrices for plant $P$:
\begin{equation}
    \begin{split}
    &   A = \left[\begin{array}{ccc}
            1.0064 & 0.0501 & 0\\
            0.2553 & 1.0064 & 0\\
            -0.0196 & -0.0005 & 1
        \end{array}\right];\ B = \left[\begin{array}{c}
            -0.00025\\
            -0.01\\
            0.02
        \end{array}\right];\\
    &   C = \left[\begin{array}{ccc}
            1 & 0 & 0\\
            0 & 0 & 1\\
        \end{array}\right];\ D = \left[\begin{array}{c}
            0\\
            0
        \end{array}\right];
    \end{split}    
\end{equation}

\section{$H_\infty$ optimal controller design}
The mean system is varying by different drop out probability $e$. Since $\E\{\xi\}= (1-e)I$. To avoid designing the optimal controller for different mean system, we do some block diagram transformations to a architecture that optimal controller/receiver change linearly w.r.t. $e$ \cite{Jeffthesis}. Use $\tilde{L}$ to denote $z^{-1} L$. 

% \begin{figure}[h]
%     \centering
%     \includegraphics[scale=0.5]{Figures/block_diagram_operation.PNG}
%     \caption{The step by step operations from figure (\ref{fig:CLblock}.a) to figure (\ref{fig:CLblock}.b)}
%     \label{fig:detailedblocksteps}
% \end{figure}


If there is no packet drop, then $e=0$, $\xi$ is perfectly connected as in figure (\ref{fig:CLblock}.c). Let $\hat{K} = K(I-\tilde{L})^{-1}$, $\hat{L} = L(I-\tilde{L})^{-1}$. It's easy to express $K$ and $L$ in terms of $\hat{K}$ and $\hat{L}$.
$$
L =\hat{L} (z^{-1} \hat{L}+I)^{-1} ;\ K = \hat{K} (z^{-1} \hat{L}+I)^{-1} .
$$
When there is $e$ probability of packet drop in each channel, see figure (\ref{fig:CLblock}.d). We can accordingly adjust the optimal controller to generate the same closed loop mean system like below,

$$
\hat{L}_e = \frac{1}{1-e} \hat{L};\ \hat{K}_e = \frac{1}{1-e} \hat{K}.
$$
\begin{equation}\label{eqn:transfrom_LK}
    L =\hat{L}_e (z^{-1} \hat{L}_e+I)^{-1} ;\ K = \hat{K}_e (z^{-1} \hat{L}_e+I)^{-1} .
\end{equation}



$[\hat{K},\hat{L}]^T$ is the controller for the system $[P,-z^{-1} I_{2}]$. So we can use $H_\infty$ optimal controller synthesis to obtain this controller. The solution of $H_\infty$ optimal synthesis is guaranteed to be proper and well-posed. From equation (\ref{eqn:transfrom_LK}), we know $L$ and $K$ are proper and well-posed whatever value $e$ takes. 

Then in the closed loop mean system, we add following noise and choose following performance output to complete the $H_\infty$ optimal synthesis, shown in figure (\ref{fig:GPlantForHinfsyn}).
\begin{figure}[h]
    \centering
    \includegraphics[scale=0.45]{figures/RMSSfigure/Generalizedplant1.PNG}
    \caption{$w$ are the noises, while $q$ are the performance outputs. $W_i$ is the weight matrix for inputs. $W_o$ is the weight matrix for outputs.}
    \label{fig:GPlantForHinfsyn}
\end{figure}
By finely tuning the weight matrix $W_i$ and $W_o$, we can obtain a satisfactory controller.
With this controller, the $H_\infty$ norm from $w_1$ to $q_1$ in figure(\ref{fig:GPlantForHinfsyn}) is $1.62$. 

\section{State space representation of CL system}
The state space representation for $[P,-z^{-1}I_2]$ is
$$
\bar{A}=\mathrm{diag}(A,\mathbf{0}_{2\times2});\ \bar{B}=\mathrm{diag}(B,I_{2})
$$
$$
\bar{C} = [C,-I_2];\ \bar{D}=[D,\mathbf{0}_{2\times2}]=\mathbf{0}_{2\times3}
$$
Now we assume controller and receiver 
$$
\left[\begin{array}{c}
    \hat{K}  \\
    \hat{L} 
\end{array}\right]
=\left[\begin{array}{c;{2pt/2pt}c}
    A_c & B_c\\ \hdashline[2pt/2pt]
    C_{c_k} & D_{c_k} \\
    C_{c_l} & D_{c_l}
\end{array}\right];\ 
C_c =\left[\begin{array}{c}
    C_{c_k}  \\
     C_{c_l} 
\end{array}\right];\ 
D_c =\left[\begin{array}{c}
    D_{c_k}  \\
    D_{c_l} 
\end{array}\right]
$$
Let $n_c$ denote the size of $A_c$. We can then construct the state space representation of figure (\ref{fig:CLRMSplant}.b)

$$
A_{cl} = \left[\begin{array}{cc}
            \bar{A}+\bar{B}D_c\bar{C} & \bar{B}C_c\\
            B_c\bar{C} & A_c 
        \end{array}\right]; \ 
B_v = \left[\begin{array}{c}
    \bar{B}D_c  \\
    B_c
\end{array}\right];\ 
B_p = \left[\begin{array}{c}
    \bar{B}(:,1)  \\
    \mathbf{0}_{n_c\times1}
\end{array}\right];
$$
$$
C_y = \left[\begin{array}{cc}
            \bar{C} & \mathbf{0}_{2\times n_c}\\
        \end{array}\right]; \ 
D_{yv} =\mathbf{0}_{2\times 2};\ D_{yp} =\mathbf{0}_{2\times 1};
$$
$$
C_q = \left[\begin{array}{cc}
            D_{c_k}\bar{C} & C_{c_k}\\
        \end{array}\right]; \
D_{qv} =D_{c_k};\ D_{qp} =1.
$$

%%%%%%%%%%%%%%%%%%%%%%%%%%%%%%%%%%%%%%%%%%%%%%%%%%%%%%%%%%%%%%%%%%%%%%%%%%%%%%%%
Care has been taken in this thesis to minimize the use of jargon and
acronyms, but this cannot always be achieved. This appendix defines
jargon terms in a glossary and contains a table of acronyms and their
meaning.
%%%%%%%%%%%%%%%%%%%%%%%%%%%%%%%%%%%%%%%%%%%%%%%%%%%%%%%%%%%%%%%%%%%%%%%%%%%%%%%%

%%%%%%%%%%%%%%%%%%%%%%%%%%%%%%%%%%%%%%%%%%%%%%%%%%%%%%%%%%%%%%%%%%%%%%%%%%%%%%%%
% Glossary {{{
%%%%%%%%%%%%%%%%%%%%%%%%%%%%%%%%%%%%%%%%%%%%%%%%%%%%%%%%%%%%%%%%%%%%%%%%%%%%%%%%

%%%%%%%%%%%%%%%%%%%%%%%%%%%%%%%%%%%%%%%%%%%%%%%%%%%%%%%%%%%%%%%%%%%%%%%%%%%%%%%%

% Table formatting

% Heading for the first page

%%%%%%%%%%%%%%%%%%%%%%%%%%%%%%%%%%%%%%%%%%%%%%%%%%%%%%%%%%%%%%%%%%%%%%%%%%%%%}}}
