\chapter{Preliminaries and Notations}
\label{prelim}
We provide most of the notations used throughout this thesis in this chapter.

\section{Sets, Fields, and Vector spaces}
The set of natural number \(\{0,1,2,\hdots\}\) is denoted by \(\mathbb{N}\). The sets of real numbers and complex numbers are denoted by \(\mathbb{R}\) and \(\mathbb{C}\). Vectors and matrices over these fields are denoted by superscripts, e.g., $\mathbb{R}^{n}$, $\mathbb{R}^{m\times n}$, and $\mathbb{C}^{m\times n}$.
Given a complex-valued \(m\times n\) matrix \(W\), \(W^\hermconj\) denotes its hermitian conjugate while \(W^\top\) denotes its transpose.
The set of real symmetric and complex Hermitian matrices of dimension $m\times m$ is denoted by $\mathbb{S}^{m}$. For a Hermitian matrix $W \in \mathbb{S}^{m}$, the notation $W > 0$ denotes positive definiteness, and $W \geq 0$ denotes positive semidefiniteness.

We denote vertical stacking by \(\ver[x_i]_{i\in\mathcal{I}}\) and block-diagonal concatenation by \(\diag[x_i]_{i\in\mathcal{I}}\), where \(\mathcal{I}\) is the index set. The subscript \(i\in\mathcal{I}\) is omitted when no ambiguity arises. 
We use \(\succeq\) to denote the element-wise compare between matrices and vectors, \(A \succeq B\) iff \(a_{ij} \geq b_{ij},\;\forall (i,j)\).

We will mainly analyze discrete-time linear systems in this thesis. The following sets are essential to discrete-time analysis. Let \(\mathbb{D}\coloneqq \{z\in\mathbb{C}:|z|<1\}\) be the open unit circle on \(\mathbb{C}\), and \(\bar{\mathbb{D}}\) be its closure. Let \(\mathcal{C}\) be the counterclockwise path on the unit circle.

\section{Linear Systems and Transfer Function Spaces}

A linear system is a mapping between vector spaces that satisfies superposition.

\begin{defn}[Linear system]
Let $\mathcal U$ and $\mathcal Y$ be vector spaces. A system
\[
G: \mathcal U \to \mathcal Y
\]
is said to be linear if for all $u_1,u_2 \in \mathcal U$ and all scalars
$\alpha,\beta$,
\[
G(\alpha u_1 + \beta u_2)
= \alpha G(u_1) + \beta G(u_2).
\]
\end{defn}
In particular, a discrete-time dynamical system can be represented as a quadruple \((A,B,C,D)\) in state-space form.
\begin{subequations}\label{eq:main}
\begin{align}
G : x[k + 1] &= A x[k] + B u[k], \label{eq:main:a} \\
y[k]     &= C x[k] + D u[k]. \label{eq:main:b}
\end{align}
\end{subequations}
The unique matrix-valued transfer function corresponding to the system is given by
\[
G(z) = D + C (z I -A)^{-1} B.
\]
The number of rows is the number of output channels, and the number of columns is the number of input channels.

\begin{defn} [Rational proper transfer function matrix]
Let $G(z)$ be an $m\times n$ transfer function matrix.  
We say that $G(z)$ is a rational proper transfer function matrix if each entry
$G_{ij}(z)$ is a rational function of $z$ and satisfies
\[
\lim_{z\to\infty} G_{ij}(z) < \infty, \quad \forall\, i=1,\ldots,m,\; j=1,\ldots,n.
\]
The set of all $m\times n$ rational proper transfer function matrices is denoted by
$\mathcal{R}_p^{\,m\times n}$. Superscripts are used to specify the dimensions of transfer function matrices when they enhance clarity and readability.
\end{defn}
\begin{rem}
Given a system's state-space realization \((A,B,C,D)\), \(G(z) \in \mathcal{R}_p\) if and only if \(A\) is a finite-dimensional matrix.
\end{rem}
Based on the transfer function matrix (TFM) value on the unit circle, we can introduce two useful classes of TFMs.
\begin{defn} \cite{Zhou1996RobustControl} 
\(\mathcal{L}_\infty(\mathcal{C})\) space is the Banach space of matrix-valued functions that are essentially bounded on \(\mathcal{C}\), with a norm defined as \(
\|G\|_{\infty} \doteq \mathrm{ess} \sup_{\omega \in[0,2\pi)} \bar{\sigma}(G(e^{j\omega}))\).
\(\mathcal{H}_\infty\) is the subspace of \(\mathcal{L}_\infty(\mathcal{C})\) that are analytic on \(\mathbb{C} \backslash \bar{\mathbb{D}}\). \(\mathcal{RH}_\infty\) is the real rational subspace of \(\mathcal{H}_\infty\), which contains all rational proper stable transfer functions.
\end{defn}
  
\begin{defn}
\(\mathcal{L}_2(\mathcal{C})\) space is the Hilbert space of matrix-valued function on \(\mathcal{C}\) and consists of all functions \(G\) with the following integral finite, i.e.
\begin{equation*}
    \frac{1}{2\pi}\int_0^{2\pi}\mathrm{trace}[G^\hermconj (e^{j\omega}) G(e^{j\omega})]d\omega < \infty, 
\end{equation*}
with an inner product \(\langle \cdot, \cdot \rangle\) defined as:
\begin{equation*}
    \langle F, G \rangle \coloneqq \frac{1}{2\pi}\int_0^{2\pi}\mathrm{trace}[F^\hermconj (e^{j\omega}) G(e^{j\omega})]d\omega.
\end{equation*}
As usual, define the Hardy space \(\mathcal{H}_2\) as the subspace of \(\mathcal{L}_2(\mathcal{C})\) that are analytic on \(\mathbb{C} \backslash \bar{\mathbb{D}}\). Define the corresponding \(\|\cdot\|_{\mathcal{H}_2}\) as:
\begin{equation}\label{eqn::H2defn}
    \|G\|_{\mathcal{H}_2} \coloneqq \sqrt{\sup_{r>1}\frac{1}{2\pi}\int_0^{2\pi}\mathrm{trace}[G^\hermconj(r e^{j\omega})G(r e^{j\omega})]d\omega}
    = \sqrt{\frac{1}{2\pi}\int_0^{2\pi}\mathrm{trace}[G^\hermconj(e^{j\omega})G(e^{j\omega})]d\omega}.
\end{equation}
The second equality holds from the Maximum Modulus Theorem \cite{Zhou1996RobustControl}. We use \(\mathcal{RH}_2\) to denote the real (coefficient) rational subspace of \(\mathcal{H}_2\).
\end{defn}
% In analogy, we can define \(\mathcal{L}_\infty(\mathcal{C})\), \(\mathcal{H}_\infty\) and \(\mathcal{RH}_\infty\).
\begin{rem}
\(\mathcal{RH}_\infty = \mathcal{RH}_2\) in the discrete-time context, since both are equivalent to having a finite number of poles and all in \(\mathbb{D}\).     
\end{rem}

Throughout this thesis, regular capital letters denote matrices, transfer function matrices, or their impulse response sequences when unambiguous; when clarification is needed, a hat \(\hat{\cdot}\) is used to distinguish a transfer function (matrix) from its impulse response sequence.

\section{System interconnections}
We use \(\mathcal{F}_l/\mathcal{F}_u\) to denote the lower/upper linear fractional transformation (LFT) between two linear systems. 
\begin{defn}
Let \(G\) be a proper transfer function matrix partitioned as
\[
G(z) = \begin{bmatrix}
    G_{11}(z) & G_{12}(z) \\
    G_{21}(z) & G_{22}(z)
\end{bmatrix} \in \mathcal{R}_p^{(m_1 + m_2)\times (n_1 +n_2)},
\]
and let \(Q_l(z) \in \mathcal{R}_p^{n_2 \times m_2}\) and \(Q_u(z) \in \mathcal{R}_p^{n_1 \times m_1}\) be two other transfer function matrix. Then we can formally define a lower LFT with respect to \(Q_l\) as:
\[
\mathcal{F}_l(G,Q_l) \doteq G_{11} + G_{12} Q_l (I - G_{22} Q_l)^{-1} G_{21}
\]
provided that \(I - G_{22}(\infty) Q_l(\infty)\) is non-singular.

In analogy, we can define an upper LFT with respect to \(Q_u\) as:
\[
\mathcal{F}_u(G,Q_u) \doteq G_{22} + G_{21} (I - Q_u G_{11})^{-1} Q_u G_{12}
\]
provided that \(I - Q_u(\infty) G_{11}(\infty) \) is non-singular.
\end{defn}

Notably, the feedback interconnection between two systems with compatible dimensions is a special form of \(\mathcal{F}_l\):
\[
H = \mathrm{feedback}(H^1,H^2) \coloneqq \mathcal{F}_l(\begin{bmatrix}
    I \\
    I
\end{bmatrix} H^1 
\begin{bmatrix}
    I & I
\end{bmatrix}, H^2).
\]


\section{Time-Domain Signals and Operator Representations}
A linear-time-invariant (LTI) system (not necessarily finite-dimensional) can be equivalently characterized by its unilateral infinite impulse responses in the time domain. The impulse responses of discrete-time systems produce a matrix sequence defined on \(\mathbb{N}\).


\(\ell_2^{m\times n}[\mathbb{Z}]\) denotes the Hilbert space of sequences of \(m\times n\) complex-valued matrices, with inner product defined as
\[
\langle H,G\rangle = \sum_{k=-\infty}^{\infty} \mathrm{trace}(G^\hermconj[k]H[k]).
\]



The \(\ell_2^{m\times n}[\mathbb{Z}]\) space can be decomposed as the direct sum of two spaces of sequences \(\ell_2^{m\times n}[\mathbb{N}] \oplus \ell_2^{m\times n}[\mathbb{Z}^-]\).
The unilateral z-transform of \(G\in \ell_2^{m\times n}[\mathbb{N}]\) is
\[
\hat{G}(z) = \sum_{k=0}^\infty G[k] z^{-k}.
\]
Since we will mostly focus on the positive axis unilateral matrix sequences, we abbreviate \(\ell_2^{m\times n}[\mathbb{N}]\) as \(\ell_2^{m\times n}\). 
Given any transfer function matrix (TFM) \(\hat{G}:\mathbb{C}\mapsto \mathbb{C}^{m\times n}\), the inverse Z-transform is
\[
\mathcal{Z}^{-1}:G[n] = \frac{1}{2\pi j} \oint_\mathcal{C} \hat{G}(z)z^{n-1} dz, 
\]
For any \(\hat{G}\in\mathcal{H}_2^{m\times n}\), its inverse Z-transform gives a sequence \(\{G\}\in\ell_2^{m\times n}\).
Furthermore, given a weight matrix \(W \in \mathbb{S}^{m} \ge 0\), for a matrix \(M \in \mathbb{C}^{m \times n}\) we define
\(\|M\|_W^2 \coloneqq \mathrm{trace}(M^\hermconj W M)\). More generally, for a sequence \(\{G\}\in\ell_2^{m\times n}\), the \(W\)-weighted \(\ell_2\) norm is defined as
\[
\|G\|_W^2 \coloneqq \sum_{i=0}^{\infty} \mathrm{trace}(G[i]^\hermconj W G[i]).
\]
Especially, the trace operator can be omitted when \(G\) is a sequence of vectors.

% We may drop the superscripts later when no ambiguity arises. 


% {\color{red}Throughout this paper, \(\hat{G}\) denotes a transfer function, while the regular capital letter \(G = \mathcal{Z}^{-1}(\hat{G})\) denotes its associated impulse response sequence.}

% {\color{red}Generating IIR from state-space models.}

\begin{defn}
Given an LTI MIMO (\(m\times n\)) transfer function matrix \(\hat{G}\), its \(\ell_1\) norm is defined as
\begin{equation}
    \|\hat{G}\|_1 \triangleq \|G\|_1 =  \underset{1\leq i\leq m}{\max} \sum_{j=1}^n \sum_{t=0}^\infty |G_{ij}[t]|,
\end{equation}
Furthermore, \(\ell_1^{m\times n}\) defines the set of all \(m\times n\) matrix sequences \(G\) such that \(\|G\|_1 < \infty\). \(\hat{G}\) is said to be in \(\ell_1^{m\times n}\) if its impulse sequence \(G\) is in \(\ell_1^{m\times n}\).
\end{defn}
\begin{rem}
Given any \(\hat{G}\in \mathcal{RH}_\infty^{m\times n}\), its impulse response \(G = \mathcal{Z}^{-1}(\hat{G}) \in \ell_1^{m\times n}\).
\end{rem}
\begin{rem}
The $\ell_1$ norm of a linear operator $\hat{G}$ can be interpreted as the induced peak-to-peak norm:
\begin{equation}
    \|\hat{G}\|_{1} \;=\; \sup_{\|w\|_{\infty} \leq 1} \;\|\hat{G}w\|_{\infty}.
\end{equation}
\end{rem}
% For any Banach space \(X\) and its non-empty closed convex subspace \(C\), we use \(\mathbb{P}_C x\) to denote the projection of \(x\in X\) onto \(C\).


We also introduce the following truncation operators on the time domain responses for preparation.
Let \(\ell^{m\times n}\) be the all sequences of \(m \times n\) matrices, \(x\in\ell^{m\times n}\).
For any fixed integer \(N\in \mathbb{N}\), \(\mathcal{H}_N \coloneqq \{0,1,\hdots,N-1\}\), \(\mathbb{P}_N:\ell^{m\times n} \mapsto \mathbb{R}^{m\times n}\times \mathcal{H}_N\) denotes the \(N\)-horizon truncation operator:
\[
\mathbb{P}_N x = 
\left[x[0],x[1],\hdots,x[N-1]\right].
\]
\(\mathbb{T}_N:\ell^{m\times n} \mapsto \ell^{m\times n}\) denotes the tail operator:
\[
\mathbb{T}_N x = 
\left[x[N],x[N+1],\hdots\right].
\]

We let \(y\in \mathbb{R}^{m\times n} \times \mathcal{H}_N\) be an arbitrary \(N\) steps sequence. Define the following operator that pads an infinite zeros at the end and extends \(y\) into an infinite sequence, 

\(\bar{\mathbb{P}}_N: \mathbb{R}^{m\times n} \times \mathcal{H}_N \mapsto \ell^{m\times n}\)
\[
\bar{\mathbb{P}}_N y = 
\left[y[0],\hdots,y[N-1],\mathbf{0},\hdots\right].
\]
\(\bar{\mathbb{T}}_N:\ell^{m\times n} \mapsto \ell^{m\times n}\) denotes the operator adding \(N\) zeros at the beginning:
\[
\bar{\mathbb{T}}_N x = 
\left[\mathbf{0},\hdots,\mathbf{0},x[0],x[1],\hdots\right].
\]
\begin{prop}
We have some properties of these operators:
\[
\bar{\mathbb{P}}_N\mathbb{P}_N x +\bar{\mathbb{T}}_N\mathbb{T}_N x =x;\quad \mathbb{T}_N \bar{\mathbb{T}}_N x = x;\quad \bar{\mathbb{T}}_N (\mathcal{Z}^{-1} \hat{x}) = \mathcal{Z}^{-1}(\frac{1}{z^N} \hat{x}). 
\]
  
\end{prop}

% The dual of a Banach space \(\cal{X}\) is denoted by \(\mathcal{X}^*\). Given \(\mathcal{A}\) a bounded linear operator from \(\cal{X}\) to \(\mathcal{Y}\). \(\mathcal{A}^*\colon \mathcal{Y}^* \mapsto \mathcal{X}^*\) denotes its adjoint.

\section{Graph-Theoretic Notation for Networked Systems}