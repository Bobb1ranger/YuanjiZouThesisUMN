%%%%%%%%%%%%%%%%%%%%%%%%%%%%%%%%%%%%%%%%%%%%%%%%%%%%%%%%%%%%%%%%%%%%%%%%%%%%%%%
% intro.tex: Introduction to the thesis
%%%%%%%%%%%%%%%%%%%%%%%%%%%%%%%%%%%%%%%%%%%%%%%%%%%%%%%%%%%%%%%%%%%%%%%%%%%%%%%%
\chapter{Introduction}
\label{intro_chapter}
%%%%%%%%%%%%%%%%%%%%%%%%%%%%%%%%%%%%%%%%%%%%%%%%%%%%%%%%%%%%%%%%%%%%%%%%%%%%%%%%

% \begin{itemize}

% %\item Chapter 1 introduces the analytic goals pursued in this thesis.

% \item Chapter 2 briefly presents the history of, and science behind, the
% subjects presented in this thesis.

% \item In Chapter 3 the experiment is outlined.

% \item Chapter 4 describes the simulation process used in the analysis.

% \item Chapter 5 follows the chain of reconstruction software used to obtain
% meaningful results from data.

% \item Chapter 6 hashes out the strategy for analysis and presents the data and
% simulated sets that will be used in the analysis.

% \item Chapter 7 demonstrates the implementation of the event selection
% processes.

% \item In Chapter 8 those events selected in Chapter 7 are analyzed.

% \item Chapter 9 presents a final discussion of the analyses presented in the
% thesis.

% \end{itemize}
\section{Introduction}
With the advent of networks in modern days~\cite{ge2017OverviewNCS}, the system to be controlled usually comprises many subsystems spanning a vast physical expanse. Designing a controller composed of multiple communicating sub-controllers, known as a network-distributed controller, is a pertinent topic in control theory. 

Although the objectives of distributed control vary across different contexts, people have devoted substantial effort to establishing the fundamental theory of optimal distributed control for linear time-invariant (LTI) networked systems.
The goal shifts to finding a stabilizing feedback controller that optimizes closed-loop performance within specific structures, reflecting the communication limitations imposed by the network topology. These structures usually include delays and sparsities in the controller's aggregate transfer function matrices.

Searching for stabilizing controllers with an arbitrary structure is not a convex problem in general~\cite{Tsitsiklis1985NP_decentralized}. Optimal controllers for an LTI plant in some structures can even be non-linear~\cite{ Witsenhausen1968ACI}. While the Youla parametrization enables us to search over stabilizing controllers convexly through closed-loop (CL) maps \cite{YoulaPar} when there is no structure, the structural constraints on the controllers may be projected into nonconvex constraints on the Youla Parameter, as their convexity is not preserved under linear fractional transformation (LFT).


Following the Youla parameterization approach, Voulgaris is the first to notice the usefulness of constructing a doubly coprime factorization (DCF) consistent with the desired structure, which enables a direct search for the Youla parameter \(Q\) within the same structured set \cite{Voulgaris2001convexity}. Other works identify several particular structures where consistently structured DCFs are accessible and establish convex control synthesis methods for such structures~\cite{Voulgaris2000nested, XinQ2004Nested, Bamieh2005CvxSI}. Later,~\cite{SanjayLQIsufficient} introduces the notion of Quadratic Invariance (QI) to unify all the structures for which one can search for the controllers convexly~\cite{LessardQI}. As part of their contribution, they establish the theoretical foundation for parameterizing all in-structure stabilizing controllers using a stable, structured, and stabilizing initial controller. However, obtaining such an initial controller is generally nontrivial.


To address this limitation, \cite{SabauVectorization} seeks to establish the classical Youla parameterization without relying on structured DCFs. In return, they introduce an affine constraint rather than a direct structural constraint as the price for eschewing structured DCFs:
\begin{equation}\label{eqn:strcst_0}
    H - U Q V \in \mathcal{S}.
\end{equation}
This constraint does not necessarily admit finite-impulse-response (FIR) solutions for the Youla parameter, which poses a major obstacle to its use in the controller synthesis, where FIR approximation is the standard practice for solving convex model-matching problems. Nevertheless, Lamperski and Doyle show that the constraint admits FIR solutions when the structural restrictions vanish after finitely many steps, which corresponds to strongly connected networks. They exploit this property to solve the optimal distributed \(\mathcal{H}_2\) control problem via a finite-dimensional convex optimization formulation \cite{Lamperski2015H2}.


A class of recent methods, e.g., SLP~\cite{SLSparametrization, wang2019SLS}, IOP~\cite{Furieri2019IOP}, and mixed parameterizations~\cite{Zheng2022SLP_IOP}, abandon the Youla parameterization approach. These methods over-parametrize the search spaces (not just $Q$) and enforce stabilization implicitly through infinite-dimensional equality constraints, which unfortunately are not guaranteed to have finite-dimensional feasible solutions. \cite{Matni2017SLStol} later addresses the tolerance of equality constraint violations in the SLP method. Overall, their synthesis formulation is sophisticated and leads to order inflation in the controller order. Fundamentally, these new parametrizations are equivalent to the Youla Parameterization~\cite{Zheng2021Equi_SLP_IOP, Tseng2021Equi_Par}.


The Youla operator state-space (YOSS) framework provides an alternative, over-parameterizing representation for all stabilizing controllers with a predefined structure~\cite{naghnaeian2019youla}. This approach introduces the notion of a generalized Luenberger observer with internal dynamics designed for networked systems. The paper establishes a distributed analogue of the Separation Principle, showing that distributed state estimation can be carried out independently and then used together with “state-like” controllers to obtain the set of all output feedback stabilizing controllers with structure. A key advantage of this approach is that the estimator and state-like feedback controller satisfy relaxed infinite-dimensional "inequalities", which admit finite-dimensional feasible solutions.







% A norm minimization problem subject to an equation constraint is obtained. The violation is penalized in the total cost to enforce the structure and stabilization constraints.





\subsection{Contribution}
Unlike recent approaches that over-parameterize stabilizing controllers, we take a step back to focus on algebraically analyzing the constraint presented in (\ref{eqn:strcst_0}). Based on our analysis, we present a complete solution of the optimal distributed \(\mathcal{H}_2\) controller for the networked plant in this article. The key points of our contributions are:

% \begin{enumerate}
%     \item We demonstrate that quadratic invariance (QI) is insufficient for differentiating structured yet centralized controllers from network-implementable distributed controllers with a concrete example. We focus on a large subclass of QI problems where the network implementability is guaranteed.
%     \item We formulate a complete characterization of admissible controllers with a fictitious dynamical system constructed via standard Youla Parameterization. We demonstrate the equivalence between the open-loop stabilizability of this system and the feasibility of the structured stabilization problem. Furthermore, we provide a condition that determines the stabilizability.
%     \item We construct the finite-dimensional \(\mathcal{H}_2\) distributed controller algebraically from the fictitious system by solving three discrete-time algebraic Riccati equations (DARE), thus eschewing approximation methods. This result hugely reduces the computational time and the controller order.
% \end{enumerate}
\begin{enumerate}
    \item We derive an exact, finite-dimensional distributed \(\mathcal{H}_2\) controller by extending the centralized solution with an additional discrete-time algebraic Riccati equation (DARE) and a backward recursion. Solving this DARE quantifies the minimal performance degradation induced by the structural constraints and simultaneously yields the optimal controller within the prescribed structure.
    \item We give a constructive approach to find exact solutions to the constraints on Youla parameter \(Q\), i.e. \(H -U Q V \in \mathcal{S}\), which ensures that the stabilizing controller lies in \(\mathcal{S}\). This approach is broadly applicable to other structured controller synthesis problems.
    \item We provide a condition that checks the existence of stabilizing controllers in the prespecified structure and establish its connection with the standard PBH test. 
    \item We show with an example that QI alone cannot distinguish structured centralized from networked distributed controllers. 
\end{enumerate}


\section{Mathematical Notations}
We focus on the discrete-time problems in this paper. Let \(\mathbb{D}\coloneqq \{z\in\mathbb{C}:|z|<1\}\), and \(\bar{\mathbb{D}}\) be its closure. 
\(\mathcal{C}\) is the counter-clock-wise path on the unit circle.  
\begin{defn}
\cite{Zhou1996RobustControl} \(\mathcal{L}_2(\mathcal{C})\) space is the Hilbert space of matrix-valued function on \(\mathcal{C}\) and consists of all functions \(G\) with the following integral finite, i.e.
\begin{equation*}
    \frac{1}{2\pi}\int_0^{2\pi}\mathrm{trace}[G^\hermconj (e^{j\omega}) G(e^{j\omega})]d\omega < \infty, 
\end{equation*}
with an inner product \(\langle \cdot, \cdot \rangle\) defined as:
\begin{equation*}
    \langle F, G \rangle \coloneqq \frac{1}{2\pi}\int_0^{2\pi}\mathrm{trace}[F^\hermconj (e^{j\omega}) G(e^{j\omega})]d\omega.
\end{equation*}
As usual, define the Hardy space \(\mathcal{H}_2\) as the subspace of \(\mathcal{L}_2(\mathcal{C})\) that are analytic on \(\mathbb{C} \backslash \bar{\mathbb{D}}\). Define the corresponding \(\|\cdot\|_{\mathcal{H}_2}\) as:
\begin{equation}\label{eqn::H2defn}
    \|G\|_{\mathcal{H}_2} \coloneqq \sqrt{\sup_{r>1}\frac{1}{2\pi}\int_0^{2\pi}\mathrm{trace}[G^\hermconj(r e^{j\omega})G(r e^{j\omega})]d\omega}
    = \sqrt{\frac{1}{2\pi}\int_0^{2\pi}\mathrm{trace}[G^\hermconj(e^{j\omega})G(e^{j\omega})]d\omega}.
\end{equation}
The second equality holds from the Maximum Modulus Theorem \cite{ZhouRobustControl}. We use \(\mathcal{RH}_2\) to denote the real (coefficient) rational subspace of \(\mathcal{H}_2\), which contains all rational proper stable transfer functions.
\end{defn}
% In analogy, we can define \(\mathcal{L}_\infty(\mathcal{C})\), \(\mathcal{H}_\infty\) and \(\mathcal{RH}_\infty\).
\begin{defn}
\(\mathcal{L}_\infty(\mathcal{C})\) space is the Banach space of matrix-valued functions that are essentially bounded on \(\mathcal{C}\), with a norm defined as \(
\|G\|_{\infty} \doteq \mathrm{ess} \sup_{\omega \in[0,2\pi)} \bar{\sigma}(G(e^{j\omega}))\).
\(\mathcal{H}_\infty\) is the subspace of \(\mathcal{L}_\infty(\mathcal{C})\) that are analytic on \(\mathbb{C} \backslash \bar{\mathbb{D}}\). \(\mathcal{RH}_\infty\) is the real rational subspace of \(\mathcal{H}_\infty\).
\end{defn}

\(\mathcal{RH}_\infty = \mathcal{RH}_2\) for discrete time. We let \(\mathcal{R}_p\) represent the set of all rational proper transfer functions in the \(z\)-domain. Superscripts are used to specify the dimensions of transfer function matrices when they enhance clarity and readability.

We denote vertical stacking by \(\ver[x_i]_{i\in\mathcal{I}}\) and block-diagonal concatenation by \(\diag[x_i]_{i\in\mathcal{I}}\), where \(\mathcal{I}\) is the index set. The subscript \(i\in\mathcal{I}\) is omitted when no ambiguity arises. 
We use \(\succeq\) to denote the element-wise compare between matrices and vectors, \(A \succeq B\) iff \(a_{ij} \geq b_{ij},\;\forall (i,j)\).
We use \(\mathcal{F}_l/\mathcal{F}_u\) to denote the lower/upper linear fractional transformation between two systems. The feedback interconnection between two systems with compatible dimensions is a special form of \(\mathcal{F}_l\):
\[
H = \mathrm{feedback}(H^1,H^2) \coloneqq \mathcal{F}_l(\begin{bmatrix}
    I \\
    I
\end{bmatrix} H^1 
\begin{bmatrix}
    I & I
\end{bmatrix}, H^2).
\]

In this paper, we focus on discrete-time problems. Let \(\mathbb{D}\coloneqq \{z\in\mathbb{C}:|z|<1\}\), and \(\bar{\mathbb{D}}\) be its closure. 
\(\mathcal{C}\) is the counter-clock-wise path on the unit circle. In the time domain, the impulse responses of discrete-time systems produce a matrix sequence defined on \(\mathbb{N} = \{0,1,2,\hdots\}\).

Given a complex-valued \(m\times n\) matrix \(A\), \(A^\hermconj\) denotes its hermitian conjugate.
\(\ell_2^{m\times n}[\mathbb{Z}]\) denotes the Hilbert space of sequences of \(m\times n\) complex-valued matrices, with inner product defined as
\[
\langle H,G\rangle = \sum_{k=-\infty}^{\infty} \mathrm{trace}(G^\hermconj[k]H[k]).
\]
The \(\ell_2^{m\times n}[\mathbb{Z}]\) space can be decomposed as the direct sum of two spaces of sequences \(\ell_2^{m\times n}[\mathbb{N}] \oplus \ell_2^{m\times n}[\mathbb{Z}^-]\).
The unilateral z-transform of \(G\in \ell_2^{m\times n}[\mathbb{N}]\) is
\[
\hat{G}(z) = \sum_{k=0}^\infty G[k] z^{-k}.
\]
Since we will mostly focus on the positive axis unilateral matrix sequences, we abbreviate \(\ell_2^{m\times n}[\mathbb{N}]\) as \(\ell_2^{m\times n}\). 
Given any transfer function matrix (TFM) \(\hat{G}:\mathbb{C}\mapsto \mathbb{C}^{m\times n}\), the inverse Z-transform is
\[
\mathcal{Z}^{-1}:G[n] = \frac{1}{2\pi j} \oint_\mathcal{C} \hat{G}(z)z^{n-1} dz, 
\]
We define \(\mathcal{H}_2^{m\times n}\) as the set of all these matrix-value functions that are analytic on \(\mathbb{C}\backslash\mathbb{D}\). \(\mathbb{D}=\{z:|z|<1\}\). 
For any \(\hat{G}\in\mathcal{H}_2^{m\times n}\), its inverse Z-transform gives a sequence \(\{G\}\in\ell_2^{m\times n}\).
We may drop out the superscripts later when no ambiguity arises. 


We use \(\mathcal{R}_p\) to denote the set of all rational proper transfer functions in the \(z\)-domain and \(\mathcal{RH}_\infty\) to denote the set of all rational proper stable transfer functions. \(\mathcal{RH}_\infty = \mathcal{RH}_2\) in the discrete-time context, since both are equivalent to having a finite number of poles and all in \(\mathbb{D}\). 
% {\color{red}Throughout this paper, \(\hat{G}\) denotes a transfer function, while the regular capital letter \(G = \mathcal{Z}^{-1}(\hat{G})\) denotes its associated impulse response sequence.}
Throughout this paper, regular capital letters denote matrices, transfer function matrices, or their impulse response sequences when unambiguous; when clarification is needed, a hat \(\hat{\cdot}\) is used to distinguish a transfer function (matrix) from its impulse response sequence.
% {\color{red}Generating IIR from state-space models.}

\begin{defn}
Given an LTI MIMO (\(m\times n\)) transfer function \(\hat{G}\), its \(\ell_1\) norm is defined as
\begin{equation}
    \|\hat{G}\|_1 \triangleq \|G\|_1 =  \underset{1\leq i\leq m}{\max} \sum_{j=1}^n \sum_{t=0}^\infty |G_{ij}[t]|,
\end{equation}
Furthermore, \(\ell_1^{m\times n}\) defines the set of all \(m\times n\) matrix sequences \(G\) such that \(\|G\|_1 < \infty\). \(\hat{G}\) is said to be in \(\ell_1^{m\times n}\) if its impulse sequence \(G\) is in \(\ell_1^{m\times n}\).
\end{defn}
\begin{rem}
Given any \(\hat{G}\in \mathcal{RH}_\infty^{m\times n} (= \mathcal{RH}_2^{m\times n})\), its impulse response \(G = \mathcal{Z}^{-1}(\hat{G}) \in \ell_1^{m\times n}\).
\end{rem}
\begin{rem}
The $\ell_1$ norm of a linear operator $\hat{G}$ can be interpreted as the induced peak-to-peak norm:
\begin{equation}
    \|\hat{G}\|_{1} \;=\; \sup_{\|w\|_{\infty} \leq 1} \;\|\hat{G}w\|_{\infty}.
\end{equation}
\end{rem}
% For any Banach space \(X\) and its non-empty closed convex subspace \(C\), we use \(\mathbb{P}_C x\) to denote the projection of \(x\in X\) onto \(C\).
Given a weight matrix \(W \in \mathbb{S}^{n} \succ 0\), for a vector \(z \in \mathbb{R}^n\) we define
\(\|z\|_W^2 \coloneqq z^\top W z\). More generally, for a sequence \(z \in \ell_2^{n}\), the weighted \(\ell_2\) norm is defined as
\[
\|z\|_W^2 \coloneqq \sum_{i=0}^{\infty} z[i]^\top W z[i].
\]


Since \(\ell_1\) problems are more relevant to the time domain responses, we introduce the following truncation operators for preparation. Let \(\ell^{m\times n}\) be the all sequences of \(m \times n\) matrices, \(x\in\ell^{m\times n}\).
For any fixed integer \(N\in \mathbb{N}\), \(\mathcal{F}_N \coloneqq \{0,1,\hdots,N-1\}\), \(\mathbb{P}_N:\ell^{m\times n} \mapsto \mathbb{R}^{m\times n}\times \mathcal{F}_N\) denotes the \(N\) level truncation operator:
\[
\mathbb{P}_N x = 
\left[x[0],x[1],\hdots,x[N-1]\right].
\]
\(\mathbb{T}_N:\ell^{m\times n} \mapsto \ell^{m\times n}\) denotes the tail operator:
\[
\mathbb{T}_N x = 
\left[x[N],x[N+1],\hdots\right].
\]

We let \(y\in \mathbb{R}^{m\times n} \times \mathcal{F}_N\) be an arbitrary \(N\) steps sequence. Define the following operator that pads an infinite zeros at the end and extends \(y\) into an infinite sequence, 

\(\bar{\mathbb{P}}_N: \mathbb{R}^{m\times n} \times \mathcal{F}_N \mapsto \ell^{m\times n}\)
\[
\bar{\mathbb{P}}_N y = 
\left[y[0],\hdots,y[N-1],\mathbf{0},\hdots\right].
\]
\(\bar{\mathbb{T}}_N:\ell^{m\times n} \mapsto \ell^{m\times n}\) denotes the operator adding \(N\) zeros at the beginning:
\[
\bar{\mathbb{T}}_N x = 
\left[\mathbf{0},\hdots,\mathbf{0},x[0],x[1],\hdots\right].
\]
\begin{prop}
We have some properties of these operators:
\[
\bar{\mathbb{P}}_N\mathbb{P}_N x +\bar{\mathbb{T}}_N\mathbb{T}_N x =x;\quad \mathbb{T}_N \bar{\mathbb{T}}_N x = x;\ 
\]
\[
\bar{\mathbb{T}}_N (\mathcal{Z}^{-1} \hat{x}) = \mathcal{Z}^{-1}(\frac{1}{z^N} \hat{x}).
\]    
\end{prop}

The dual of a Banach space \(\cal{X}\) is denoted by \(\mathcal{X}^*\). Given \(\mathcal{A}\) a bounded linear operator from \(\cal{X}\) to \(\mathcal{Y}\). \(\mathcal{A}^*\colon \mathcal{Y}^* \mapsto \mathcal{X}^*\) denotes its adjoint.