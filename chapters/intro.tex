%%%%%%%%%%%%%%%%%%%%%%%%%%%%%%%%%%%%%%%%%%%%%%%%%%%%%%%%%%%%%%%%%%%%%%%%%%%%%%%
% intro.tex: Introduction to the thesis
%%%%%%%%%%%%%%%%%%%%%%%%%%%%%%%%%%%%%%%%%%%%%%%%%%%%%%%%%%%%%%%%%%%%%%%%%%%%%%%%
\chapter{Introduction}
\label{intro_chapter}
%%%%%%%%%%%%%%%%%%%%%%%%%%%%%%%%%%%%%%%%%%%%%%%%%%%%%%%%%%%%%%%%%%%%%%%%%%%%%%%%

% \begin{itemize}

% %\item Chapter 1 introduces the analytic goals pursued in this thesis.

% \item Chapter 2 briefly presents the history of, and science behind, the
% subjects presented in this thesis.

% \item In Chapter 3 the experiment is outlined.

% \item Chapter 4 describes the simulation process used in the analysis.

% \item Chapter 5 follows the chain of reconstruction software used to obtain
% meaningful results from data.

% \item Chapter 6 hashes out the strategy for analysis and presents the data and
% simulated sets that will be used in the analysis.

% \item Chapter 7 demonstrates the implementation of the event selection
% processes.

% \item In Chapter 8 those events selected in Chapter 7 are analyzed.

% \item Chapter 9 presents a final discussion of the analyses presented in the
% thesis.

% \end{itemize}
\section{Centralized control and network distributed control}
In systems and control, centralized control has long served as a benchmark paradigm for regulating systems of varying complexity, owing to its optimality in performance and streamlined design procedures. However, its dependence on a single, powerful central controller introduces a vulnerability to single-point failures, which poses a fundamental risk to system resilience, especially for large-scale systems in which multiple subsystems spanning a vast spatial scale are coupled.

To mitigate this vulnerability, decision-making is often decentralized across multiple agents. However, in the absence of proper coordination mechanisms, such decentralization can neglect coupling dynamics among subsystems, potentially leading to instability and, more generally, to significant degradation of global performance.

The fundamental challenge lies in the limited information available to individual system components under decentralized architectures. When subsystems are coupled through dynamics or shared performance objectives, purely local information is generally insufficient to ensure stability or achieve satisfactory system-level performance. This motivates the introduction of information exchange to enable coordination without reverting to fully centralized control.

With the advent and rapid development of communication technologies, modern networks enable fast and reliable information exchange among physically distributed subsystems~\cite{ge2017OverviewNCS}. This capability has stimulated extensive research on network-distributed controller design, in which information sharing is exploited to improve control performance in large-scale systems with localized decision-making. The paradigm of distributed control has been widely applied across disciplines, including collaborative multi-robot systems,  distributed computing, distributed transportation networks, distributed electrical power systems, smart grids, and remote surgery.

\section{Challenges in network distributed control}

While communication enhances system resilience and coordination capabilities, it also introduces significant challenges in controller design. Designers must explicitly account for communication constraints, such as limited bandwidth, transmission delays, quantization, and the associated computational burden. Moreover, during validation and implementation, additional analysis is required to assess robustness against adverse network effects, including packet dropouts, channel fading, and communication noise.


% Such systems may consist of multiple subsystems that are coupled not only through their physical dynamics, but also through shared performance objectives or cost functions.

\section{Linear-time-invariant(LTI) distributed networked systems}

While the objectives of distributed control and their respective approaches vary across application contexts, researchers have devoted substantial effort to establishing the fundamental theory of optimal distributed control for linear time-invariant (LTI) networked systems.






\section{Optimal network distributed control}

The goal shifts to finding a stabilizing feedback controller that optimizes closed-loop performance within specific structures, reflecting the communication limitations imposed by the network topology. These structures usually include delays and sparsities in the controller's aggregate transfer function matrices.

\section{Robustness of distributed control}

 
\cite{Meansquarenorm} has first proposed a robust control framework for MIMO LTI systems subject to IID random gains. A random gain is represented by its mean value (corresponding to the nominal parameter value) and a zero-mean stochastic perturbation $\Delta$ of bounded variance. \cite{Meansquarenorm} derives a necessary and sufficient condition for Mean Square Stability (MSS), showing that MSS can be equivalently thought of as robust mean stability. The framework has been extended to various network control settings \cite{padmasola2006tradeoffs, Padma06ACC, Jeff11, Matt2014cvxMScontrol, Matt2014cvxMScontrol_act, MattACC15, Matt2017thesis} which are relevant in this paper. In particular, \cite{Ma2015} is the first to show that MS performance is equivalent to an augmented MSS problem in the same vain of the analogous robust performance problems for deterministic uncertainty, and \cite{Matt2017thesis} show that the optimal MS performance synthesis is convex for a large class of systems. 



\cite{Willems1971FreqMSS, Willems1973MSstability} found state space and frequency domain necessary and sufficient conditions for MS stability. Recently, some researchers have observed similarities between robust stability and stochastic stability and have designed MS stabilizing controllers for stochastic systems using robust control methods~\cite{RobustControl_Mulnoise}.  


% A norm minimization problem subject to an equation constraint is obtained. The violation is penalized in the total cost to enforce the structure and stabilization constraints.

\subsection{Example}

Consider a two-agent distributed system:

\[
\begin{bmatrix}
    x_1[k + 1] \\ x_2[k + 1] 
\end{bmatrix} =
A \begin{bmatrix}
    x_1[k] \\ x_2[k] 
\end{bmatrix}
+ B_w \begin{bmatrix}
    w_1[k] \\ w_2[k] 
\end{bmatrix}
\]
On step \(k\) at agent 1, the most recent information of \(x_1[k]\) and history states are fully available, while the delayed information of state \(x_2[k - 1]\) and history states are available. The core problem is how to establish the MMSE estimator \(\hat{x}_2[k]\) at agent 1.


\subsection{Contribution}
Unlike recent approaches that over-parameterize stabilizing controllers, we take a step back to focus on algebraically analyzing the constraint presented in (\ref{eqn:strcst_0}). Based on our analysis, we present a complete solution of the optimal distributed \(\mathcal{H}_2\) controller for the networked plant in this article. The key points of our contributions are:

% \begin{enumerate}
%     \item We demonstrate that quadratic invariance (QI) is insufficient for differentiating structured yet centralized controllers from network-implementable distributed controllers with a concrete example. We focus on a large subclass of QI problems where the network implementability is guaranteed.
%     \item We formulate a complete characterization of admissible controllers with a fictitious dynamical system constructed via standard Youla Parameterization. We demonstrate the equivalence between the open-loop stabilizability of this system and the feasibility of the structured stabilization problem. Furthermore, we provide a condition that determines the stabilizability.
%     \item We construct the finite-dimensional \(\mathcal{H}_2\) distributed controller algebraically from the fictitious system by solving three discrete-time algebraic Riccati equations (DARE), thus eschewing approximation methods. This result hugely reduces the computational time and the controller order.
% \end{enumerate}
\begin{enumerate}
    \item We derive an exact, finite-dimensional distributed \(\mathcal{H}_2\) controller by extending the centralized solution with an additional discrete-time algebraic Riccati equation (DARE) and a backward recursion. Solving this DARE quantifies the minimal performance degradation induced by the structural constraints and simultaneously yields the optimal controller within the prescribed structure.
    \item We give a constructive approach to find exact solutions to the constraints on Youla parameter \(Q\), i.e. \(H -U Q V \in \mathcal{S}\), which ensures that the stabilizing controller lies in \(\mathcal{S}\). This approach is broadly applicable to other structured controller synthesis problems.
    \item We provide a condition that checks the existence of stabilizing controllers in the prespecified structure and establish its connection with the standard PBH test. 
    \item We show with an example that QI alone cannot distinguish structured centralized from networked distributed controllers. 
\end{enumerate}


